\documentclass[11pt,a4paper]{article}

\usepackage{longtable}
\newcommand \bt{\begin{longtable}{p{0.25\textwidth}p{0.74\textwidth}}}
\newcommand \et{\end{longtable}}
 
\usepackage[pdftex,usenames,dvipsnames]{color}

\definecolor{classbg}{rgb}{0.707,0.648,0.586}
\definecolor{fieldbg}{rgb}{0.363,0.641,0.746}
\definecolor{conbg}{rgb}{0.711,0.793,0.836}
\definecolor{descriptbg}{rgb}{0.848,0.918,0.953}

\usepackage[T1]{fontenc}
\renewcommand*\familydefault{\sfdefault}

\newcommand{\hs}{\hspace{0.5cm}}

%environment for indented description
\newenvironment{di}
{\begin{flushright}
\begin{minipage}{0.95\textwidth}
\begin{description}
}
{\end{description}
\end{minipage}
\end{flushright}
}

\usepackage[hmargin=2.5cm,vmargin=2.5cm]{geometry}
\setlength{\parindent}{0.05\textwidth}

\begin{document}

\noindent
\colorbox{classbg}{\parbox{1.0\textwidth}{\Large{Class}}}
\begin{di}
\item[\large{ResultsPrinter}]\qquad\\
This class prints out the EPS, LaTex, and raw data files for a group of random walks
\end{di}
\colorbox{fieldbg}{\parbox{1.0\textwidth}{\Large{Fields}}}\vspace{0.5cm}
\bt
\hs \textbf{c} & \emph{type: CAStatic}\\
& \hs This is the main class. Storing a link to it here gives access to all the data required for printing.\\
\hs \textbf{EPSFilename = "file.eps"} & \emph{type: String}\\
& \hs All the following strings are default file names which are overwritten.\\
\hs \textbf{baseFilename = "base"} & \emph{type: String}\\
& \hs \\
\hs \textbf{latexFilename = "file.tex"} & \emph{type: String}\\
& \hs \\
\hs \textbf{cssFilename = "file.css"} & \emph{type: String}\\
& \hs \\
\hs \textbf{dataFilename = "file.dat"} & \emph{type: String}\\
& \hs \\
\hs \textbf{timeString = "00"} & \emph{type: String}\\
& \hs \\
\hs \textbf{twoPlaces} & \emph{type: static DecimalFormat}\\
& \hs This is used for printing floating point values to two decimal places.\\
\hs \textbf{dir} & \emph{type: File}\\
& \hs The directory in which to put the output files. Will be created according to the date, hour, and minute of the run.\\
\hs \textbf{resultcount} & \emph{type: static int}\\
& \hs Currently unused. Could count the number of results written.\\
\hs \textbf{maxdots = 50} & \emph{type: static int}\\
& \hs max size of histogram. Histograms of larger size will be scaled. This stops the eps files using too many pixels.\\
\hs \textbf{sf} & \emph{type: int}\\
& \hs Scale factor for histogram. Must be applied for all results in the group of walks.\\
\et
\noindent\colorbox{conbg}{\parbox{1.0\textwidth}{\Large{Constructors}}}
\begin{di}
\item[{ResultsPrinter(CAStatic orig)}]\qquad\\
Just points c to the current instance of CAStatic so that the results can be accessed.
\end{di}
\colorbox{descriptbg}{\parbox{1.0\textwidth}{\Large{Methods}}}
\begin{di}
\item[{printLaTeX(int ind,boolean multi)}]\emph{Returns void}\\
Prints out the .tex file.\\
\item[{printData()}]\emph{Returns void}\\
Print out all the position data to a plain text file.\\
\item[{printCSS2col()}]\emph{Returns void}\\
Print a css file. Not used.\\
\item[{makeEPSFilename(int ind)}]\emph{Returns void}\\
Create a unique EPS filename using the cell lineage.\\
\item[{findScaleFactor(int maxdCount)}]\emph{Returns void}\\
Calculate the scalefactor for the histogram.\\
\item[{makeFilenames()}]\emph{Returns void}\\
Create the unique directory and filenames for this group of walks using the date and time.\\
\item[{printEPSDots(int ind)}]\emph{Returns void}\\
Print out an EPS file showing all the walks taken by this cell and also a histogram showing frequency of final positions.\\
\item[{filterForLaTex(String origString)}]\emph{Returns String}\\
Filter a string so the special LaTex characters are escaped for the LaTex file.\\
\end{di}

\end{document}
